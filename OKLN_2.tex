\documentclass{article}
\usepackage[utf8]{inputenc}
\usepackage[T1]{fontenc}
\usepackage[francais]{babel} 
\usepackage{graphicx}
\usepackage{enumitem}
\usepackage{hyperref}


\begin{document}
\begin{titlepage}
\begin{center}

\includegraphics[width=0.5\textwidth]{logo-ephec.png}\\[1cm]
{\large 3TI - 2017-2018}\\[0.5cm]

{\large Réseau d'entreprise}\\[0.8cm]
%régler l'espacement entre les lignes
\newcommand{\HRule}{\rule{\linewidth}{0.5mm}}
% Title
\HRule \\[0.7cm]
%{ 
%\includegraphics[width=0.5\textwidth]{images/sharelet.png}\\
%}
{\huge \bfseries OKLN\\[0.75cm] }
{\bfseries Groupe B }\\[1cm]
\HRule \\[2.5cm]


6 Février 2018

\end{center}
\end{titlepage}
%\title{Rapport 1}
%\author{OKLN}
%\date{5 Février 2018}

%\maketitle

\section{Introduction}
Seconde journée de projet. L'infrastructure est confrontée à des problèmes dont la résolution est difficile. La configuration de pfsense semble aussi poser quelques problèmes et du retard sur le planning est constaté.

\section{Infrastructure réseau}
    \subsection{Tâches réalisées}
        Le 5 Février 2018, nous avons réalisé une infrastructure adaptée à nos besoins. Nous avions réalisé les schémas logiques et physiques correspondant. Cependant, suite à notre entretien avec M. Schalkjiwk, nous avons apporté quelques modifications à celles-ci.
        
        Nous avons commencé par mettre en place les VLANs tels que définis dans le schéma logique.
        
        La secone étape a été la configuration ip de notre réseau. Sans cela, de nombreux services ne peuvent être installés et configurés.
        
        Ensuite, nous avons mis en place le protocoles de routage OSPF et le protocole VTP  afin de simplifier l'administration de notre infrastructure.
        
    \subsection{Difficultés rencontrées}
    
    Nous avons été confronté à quelques problèmes qui ont eu pour conséquence de ralentir notre progression. Les problèmes rencontrés ont été les suivants:
    \newline
    
    \textbf{Comment mettre en place une solution de secours en cas de panne du réseau ?} 
     \newline
    
    La solution a été de rajouter une switch L3 dans l'infrastructure. Ainsi encas de panne, un composant de secours pourra assurer le fonctionnement du réseau.
    \newline
    
     \textbf{Comment répondre au manque d'interfaces disponibles ?} 
     \newline
     
     Nous avons ajouté un switch L2 supplémentaire afin d'avoir plus d'interface à disposition et ainsi augmenté la capactité du réseau.
     \newline
     
      \textbf{Comment répartir les VLANs sur les switch L3 ?} 
     \newline
     
     2 options se présentaient à nous. Nous pouvions soit répartie les VLANs sur les 2 switch L3 afin de partager les ressources, soit les installer sur le même switch.
     C'est finalement cette dernière option qui a été adoptée, le second switch L3 servira en cas de panne dans le réseau.
      \newline
     
      \textbf{Comment définir le chemin prioritaire ?} 
     \newline
     
     Pour répondre à cette question, nous nous sommes tournés vers un protocole propriétaire nommé HSRP. Ce protocole va assurer une disponibilité optimale du réseau et une résistance aux panes intéressante.
      \newline
     
      \textbf{Comment assurer la sécurité des zones principales ?} 
     \newline
     
     Nous avons décidé de mettre la TZ et la DMZ au niveau de firewall dans le but d'améliorer la sécurité du réseau.
     
     
     
    \subsection{Remarques importantes}
   

\section{Mail}

\subsection{Technologies utilisées}

 
\section{Sécurité}
    \subsection{Tâches réalisées} 
   
    \subsection{Motivation technologies/infrastructures} 
   
\section{Logistique}
   
    \subsection{Difficultés rencontrées}
    
    \subsection{Remarques importantes}
    
\section{Services internet}
    \subsection{Tâches réalisées} 
   
    \subsection{Difficultés rencontrées} 
    
    \subsection{Technologies utilisées}
   
\section{Déploiement}
    \subsection{Tâches réalisées} 
       
    \subsection{Difficultés rencontrées}
    
    \subsection{Motivation techonlogies/infrastructures}
   
    \subsection{Remarques importantes}
 
\section{Rapports \& organisation}

    \subsection{Tâches réalisées} 
   
    \subsection{Motivation des technologies/infrastructures}
   
    \subsection{Difficultés rencontrées} 
    
    \subsection{Remarques importantes}

\section{sources}

    \subsection{sécurité}
   
    \subsection{déploiement}
   
    \subsection{rapports}
       
    \subsection{Services internet}
   
\end{document}
